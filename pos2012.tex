% ------------------------------------------------------------------
\documentclass{PoS}
\usepackage{amsmath}
\usepackage{cancel}

% Laziness shortcuts
\newcommand{\al}{\ensuremath{\alpha} }
\newcommand{\be}{\ensuremath{\beta} }
\newcommand{\De}{\ensuremath{\Delta} }
\newcommand{\vev}[1]{\ensuremath{\left\langle #1 \right\rangle} }
\newcommand{\X}{\ensuremath{\!\times\!} }
\newcommand{\Sb}{\ensuremath{\cancel{S^4}} }
\newcommand{\refcite}[1]{Ref.~\cite{#1}}
\newcommand{\fig}[1]{Fig.~\ref{#1}}
\newcommand{\secref}[1]{Section~\ref{#1}}
\newcommand{\TODO}[1]{\textcolor{red}{{\bf #1}}}
% ------------------------------------------------------------------



% ------------------------------------------------------------------
\title{MCRG study of 8 and 12 fundamental flavors}
\ShortTitle{MCRG study of 8 and 12 fundamental flavors}

\author{\speaker{Gregory Petropoulos}, Anqi Cheng, Anna Hasenfratz and David Schaich \\
  Department of Physics, University of Colorado, Boulder, CO 80309 \\
  Email: \email{gregory.petropoulos@colorado.edu}
}

\abstract{
  We explore the renormalization group properties of SU(3) gauge theories with $N_f = 8$ and 12 flavors of nearly-massless fermions using the Monte Carlo Renormalization Group (MCRG) two-lattice matching technique.
  Two-lattice matching produces a step-scaling function $s_b$; for $N_f = 8$ we find $s_b$ to be clearly non-zero from weak coupling until the onset of strong-coupling lattice artifacts.
  For $N_f = 12$, in contrast, the step-scaling function exhibits a zero that indicates the presence of an infrared fixed point.
  To carry out MCRG two-lattice matching, we optimize the RG block transformation, which produces a step-scaling function that does not correspond to a unique RG $\beta$ function.
  We propose to address this issue by using the Wilson flow to eliminate the need for optimization of the RG block transformation, and present some preliminary results produced by this approach.
}

\FullConference{The 30th International Symposium on Lattice Field Theory \\
                June 24--29, 2012 \\
                Cairns, Australia}

% ------------------------------------------------------------------
\begin{document}
\section{Introduction} % Draft complete
In recent years, many lattice studies have investigated strongly-coupled gauge--fermion systems beyond QCD, exploring both the general theoretical properties of these systems as well as potential applications to physics beyond the standard model.
Some of the latest references include~\cite{Fodor:2012uu, Aoki:2012eq, Deuzeman:2012pv, Lin:2012iw}; earlier works are reviewed in \refcite{Neil:2012cb}. % Copied from eigenvalue proceedings -- check for more recent

Here we explore the renormalization group properties of SU(3) gauge theories with $N_f = 8$ and 12 nearly-massless fermions in the fundamental representation, through the Monte Carlo Renormalization Group (MCRG) two-lattice matching technique.
This is one of several complementary analyses we are currently carrying out, two more of which (studies of the Dirac eigenmode scaling and finite-temperature transitions) are discussed in other contributions to these proceedings~\cite{Hasenfratz:2012fp, Schaich:2012fr}.

One of us previously studied MCRG two-lattice matching for the 12-flavor system with almost the same nHYP-smeared staggered action that we use here~\cite{Hasenfratz:2011xn}.
As in \refcite{Hasenfratz:2011xn}, our gauge action includes both fundamental and adjoint plaquette terms, with coefficients related by $\be_A = -0.25\be_F$.
(At the perturbative level, this implies $\be_F = 12 / g^2$~\cite{Cheng:2011ic}.)
The only change between this work and \refcite{Hasenfratz:2011xn} is that we use nHYP smearing parameters are $(0.5, 0.5, 0.4)$, instead of $(0.75, 0.6, 0.3)$.
By changing the nHYP smearing parameters in this way, we are able to access stronger couplings without encountering numerical problems.
At such strong couplings, we encounter a lattice phase in which the single-site shift symmetry (``$S^4$'') of the staggered action is spontaneously broken (``$\Sb$'')~\cite{Cheng:2011ic, Schaich:2012fr}.\footnote{An interpretation of the \Sb lattice phase in terms of relevant next-to-nearest neighbor interactions was recently presented by \refcite{Deuzeman:2012ee}.}
We avoid the \Sb lattice phase in this work.

In the next section, we review the MCRG two-lattice matching procedure, and present results for the 8- and 12-flavor systems in \secref{sec:MCRG_results}.
Because the optimization required by two-lattice matching does not allow us to determine a unique RG \be function, in \secref{sec:Wflow} we propose an improved technique based on the Wilson flow, for which we present preliminary results.
% ------------------------------------------------------------------



% ------------------------------------------------------------------
\section{\label{sec:MCRG}Two-lattice matching procedure} % Draft complete
Two-lattice matching locates pairs of couplings $(\be_F, \be_F')$ for which lattice correlation lengths obey $\xi(\be_F) = 2\xi(\be_F')$.
We accomplish this by repeatedly applying RG block transformations (with scale factor $s = 2$) to lattices of volume $24^3\X48$, $12^3\X24$ and $6^3\X12$.
By blocking the larger lattices $n_b$ times and the smaller lattices only $n_b - 1$ times, we obtain blocked systems of the same lattice volume.
Matching observables on the blocked systems then immediately provides the desired relation between the correlation lengths of the original unblocked systems, and defines $\De\be_F \equiv \be_F - \be_F'$.
The specific observables we consider are the plaquette and all three six-link loops, with an eight-link loop also monitored as a check.
Each observable may predict a different $\De\be_F$; we use the spread in these results as a systematic error that dominates our uncertainties.

Following \refcite{Hasenfratz:2011xn}, our RG block transformation involves two HYP smearings with two parameters fixed and the third left free: $(\al, 0.2, 0.2)$.
We optimize the RG block transformation by tuning \al so that $\De\be$ determined by comparing $12^3\X24 \to 3^3\X6$ vs.\ $6^3\X12 \to 3^3\X6$ is the same as that determined by comparing $24^3\X48 \to 3^3\X6$ vs.\ $12^3\X24 \to 3^3\X6$.
That is, we require that consecutive RG block transformations yield the same $\De\be$.
% Figure to illustrate optimization?
We use all three lattice volumes in this optimization so that all matching is performed on the same blocked lattice volume of $3^3\X6$.
\refcite{Hasenfratz:2011xn} shows that this procedure eliminates potential finite-volume effects.
Once we have optimized the blocking, $\De\be_{opt}$ can be considered a step-scaling function $s_b$.
The step-scaling function corresponds to the discrete RG \be function with scale factor $s = 2$, in the renormalization scheme defined by the optimized RG block transformation.
% ------------------------------------------------------------------



% ------------------------------------------------------------------
\section{\label{sec:MCRG_results}MCRG results} % Draft complete
Our 8-flavor results for $s_b$ are significantly non-zero for all couplings we can explore, as shown in \fig{fig:MCRG}.
We cannot perform matching for strong couplings $\be_F < 5.4$ due to the \Sb lattice phase.
At weaker couplings (larger $\be_F$), the step-scaling function approaches the perturbative prediction $s_b \approx 0.6$.

\begin{figure}[htpb]
  \centering
  \includegraphics[width=0.45\linewidth]{8flavor.pdf}\hfill
  \includegraphics[width=0.45\linewidth]{12flavor.pdf}\hfill
  \caption{Results for the step scaling function $s_b$ from MCRG two-lattice matching with $24^3\X48$, $12^3\X24$ and $6^3\X12$ lattice volumes and the RG block transformation defined in the text, for $N_f = 8$ (left) and $N_f = 12$ (right).  The blue dashed line is the perturbative prediction for $s_b$ at asymptotically weak coupling.}
  \label{fig:MCRG}
\end{figure}

Our 12-flavor results are consistent with the presence of an infrared fixed point (IRFP).
Across a broad range $4 \leq \be_F \leq 7$, we find $s_b = 0$ within errors.
When interpreting this result, it is crucial to recall that our optimization of the RG block transformation means that we probe a different renormalized trajectory at each coupling $\be_F$.
Each different renormalized trajectory corresponds to a different renormalization scheme.
Because the location of an IRFP is scheme-dependent, if an IRFP is present, each different renormalization scheme may locate it at a different coupling.
Specifically, the optimization moves the fixed point to $\be_F$, so that $s_b = 0$.

It is potentially interesting that we do not directly observe backward flow ($s_b < 0$) in this study, even though backward flow was seen in \refcite{Hasenfratz:2011xn}, where the lattice action differs from ours only in the choice of nHYP smearing parameters.
This is in part a consequence of the \Sb lattice phase, which prevents us from exploring arbitrarily strong couplings.
However, \refcite{Hasenfratz:2011xn} observed backward flow even at relatively weak couplings.
As with $N_f = 8$, at weak coupling our 12-flavor results approach the perturbative prediction $s_b \approx 0.3$.
% ------------------------------------------------------------------



% ------------------------------------------------------------------
\section{\label{sec:Wflow}Determining a discrete \be function} % In progress
As discussed above, the optimization step in MCRG two-lattice matching has the consequence that a different RG \be function is probed for each large-volume coupling (each point in \fig{fig:MCRG}).
We conclude with a proposal to determine a single (discrete) $\beta$ function at all couplings, which will greatly simplify the presentation and interpretation of our future results.
Instead of optimizing the RG block transformation, we carry out two-lattice matching after using the Wilson flow to move the system closer to the renormalized trajectory.

The Wilson flow was recently used to compute a step-scaling function in \refcite{Fodor:2012td}; by incorporating it into an MCRG analysis we need not rely on a perturbative definition of a renormalized coupling.
Preliminary results\dots
% ------------------------------------------------------------------



% ------------------------------------------------------------------
\section*{Acknowledgments} % Draft complete
We thank D\'aniel N\'ogr\'adi for helpful comments on the Wilson flow.
This research was partially supported by the U.S.~Department of Energy (DOE) through Grant No.~DE-FG02-04ER41290 (A.~C., A.~H.\ and D.~S.) and by the DOE Office of Science Graduate Fellowship (SCGF) Program made possible by the American Recovery and Reinvestment Act of 2009.
The SCGF Program is administered for the DOE by the Oak Ridge Institute for Science and Education, managed by Oak Ridge Associated Universities under Contract No.~DE-AC05-06OR23100.
Numerical calculations were carried out on the HEP-TH and Janus clusters at the University of Colorado; at Fermilab under the auspices of USQCD supported by the DOE SciDAC program; and at the San Diego Computing Center through the Extreme Science and Engineering Discovery Environment supported by National Science Foundation (NSF) Grant No.~OCI-1053575.
Janus is supported by NSF Grant No.~CNS-0821794.
% ------------------------------------------------------------------



% ------------------------------------------------------------------
\bibliographystyle{utphys}
\bibliography{pos2012}
\end{document}
% ------------------------------------------------------------------
