\documentclass{PoS}

\title{MCRG study of 8 and 12 fundamental flavors with mixed fundamental-adjoint gauge action in strong coupling}

\ShortTitle{MCRG study of 8 and 12 fundamental flavors with mixed fundamental-adjoint gauge action in strong coupling}

\author{\speaker{Gregory Petropoulos}%
         \thanks{A footnote may follow.}\\
        University of Colorado\\
        E-mail: \email{gregory.petropoulos@colorado.edu}}

%\author{Another Author\\
%        Affiliation\\
%        E-mail: \email{...}}

\abstract{
  I explore the behavior of the beta function of SU(3) gauge theories with 8 and 12 flavors of fundamental fermions using Monte Carlo Renormalization Group (MCRG) techniques.  
}

\FullConference{The 30th International Symposium on Lattice Field Theory\\
                 June 24 – 29,  2012\\
                 Cairns, Australia}

\begin{document}

\section{Introduction}
The behavior of strongly coupled ...

Our groups MCRG analysis is part of a broader study of SU(3) gauge theories with 8 and 12 flavors of fundamental fermions.  Other characterstics of these theories that we have studied include the finite temperature phase diagram, eigenvalues, and mass spectra.  Understanding how the beta function runs is very importaint in determining basic characteristics of a theory.  Several other groups are also interested in these theories, many of whome have also calculated the step scaling function using schrodinger functional (other ways).  

\section{Two Lattice MCRG Matching}
For a more detailed description of two lattic matching see\cite{annaMCRG}.  The fundamental idea of two lattice matching is to geneate pairs of couplings $(\beta, \beta^{\prime})$ such that the lattice correlation lenght obeys $\xi(\beta)=2\xi(\beta^{\prime})$.  Here we define the step scaling function, the analogue of the RG $\beta$ function, as $s_b= \lim_{n_b\to\infty}(\Delta\beta=\beta − \beta^{\prime})$.  Two blocked actions are identical if all blocked observables' expectation values are identical.  The action matching process is summarized in two steps:

\begin{enumerate}
\item \emph{Matching:}  Match a given operator measured on two lattices blocked down to the same size $L_b$, where $L_b=\frac{L}{2^{n_b}}$
\begin{equation}
\Delta\beta(\beta;n_b,L_b)=\beta-\beta^{\prime}
\end{equation}
\begin{equation}
<O(\beta;n_b,L_b)>=<O(\beta^{\prime};n_b-1,L_b)>
\end{equation}
\item \emph{Optimization:}  Tune the blocking parameter such that consecutive steps yield the same $\Delta\beta$
\begin{equation}
\Delta\beta(\beta;n_b,L_b,\alpha_{optimal})=\Delta\beta(\beta;n_b-1,L_b,\alpha_{optimal})
\end{equation}
\end{enumerate}

To accomplish this end I generate three lattices sizes over a variety of $\beta$ values.  We used a blocking factor of two and three volume sizes to do finite volume corrected two lattice matching:  $24^3x48$, $12^3x24$, $6^3x12$.  All of our volumes were blocked down to a volume of $3^3x6$ at which point we performed matching with the $24^3x48$ and $12^3x24$ as well as the $12^3x24$ and $6^3x12$.  This gives me two values of $\Delta\beta$.  To optimize the block transformation we tune a blocking parameter $\alpha$; $\alpha_{optimal}$ is the value of $\alpha$ for which the two values of $\Delta\beta$ for the $24^3x24\to3^3x6$ matching and the $12^3x24\to3^3x6$ matching are identical.  Once we have optimized the blocking, $s_b$ can be approximated as $\Delta\beta_{optimal}$.

A great feature of MCRG is that it is fairly insensitive to finite volume effects since we are always comparing measurmetns on the same lattice size.  Because we are using expectation values of observables to compare actions this feature also allows us to work at small lattice sizes and still achieve good results.  Additionally because we are comparing local operators the statistical accuracy is generally good even with modest datasets.  Finally MCRG is a nice method for determining the step scaling function because it does not require special lattices.  The lattices used to generate this MCRG result are the same lattices that were used throughour program of study.


\section{8 Flavor Results}
Our results for 8 flavors shows a step scaling function that is running 

\section{12 Flavor Results}
Our resutls for 12 flavors 

\section{Wilson Flow}

\section{Wilson Flow and MCRG...a shotgun wedding}

\section{Conclusion}

\section{Acknowledgments}
This research was supported in part by an award from the Department of Energy (DOE) Office of Science Graduate Fellowship Program (DOE SCGF). The DOE SCGF Program was made possible in part by the American Recovery and Reinvestment Act of 2009.  The DOE SCGF program is administered by the Oak Ridge Institute for Science and Education for the DOE. ORISE is managed by Oak Ridge Associated Universities (ORAU) under DOE contract number DE-AC05-06OR23100.  All opinions expressed in this paper are the author's and do not necessarily reflect the policies and views of DOE, ORAU, or ORISE.

Cite Janus

\begin{thebibliography}{99}
  \bibitem{...} ....
\end{thebibliography}

\end{document}
