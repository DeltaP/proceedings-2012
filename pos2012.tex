% ------------------------------------------------------------------
\documentclass{PoS}
\usepackage{amsmath}
\usepackage{cancel}

% Laziness shortcuts
\newcommand{\al}{\ensuremath{\alpha} }
\newcommand{\be}{\ensuremath{\beta} }
\newcommand{\De}{\ensuremath{\Delta} }
\newcommand{\vev}[1]{\ensuremath{\left\langle #1 \right\rangle} }
\newcommand{\X}{\ensuremath{\!\times\!} }
\newcommand{\Sb}{\ensuremath{\cancel{S^4}} }
\newcommand{\MSbar}{\ensuremath{\overline{\textrm{MS}}} }
\newcommand{\refcite}[1]{Ref.~\cite{#1}}
\newcommand{\fig}[1]{Fig.~\ref{#1}}
\newcommand{\secref}[1]{Section~\ref{#1}}
\newcommand{\TODO}[1]{\textcolor{red}{{\bf #1}}}
% ------------------------------------------------------------------



% ------------------------------------------------------------------
\title{MCRG study of 8 and 12 fundamental flavors}
\ShortTitle{MCRG study of 8 and 12 fundamental flavors}

\author{\speaker{Gregory Petropoulos}, Anqi Cheng, Anna Hasenfratz and David Schaich \\
  Department of Physics, University of Colorado, Boulder, CO 80309 \\
  Email: \email{gregory.petropoulos@colorado.edu}
}

\abstract{ % Draft complete
  We explore the renormalization group properties of SU(3) gauge theories with $N_f = 8$ and 12 flavors of nearly-massless fermions using the Monte Carlo Renormalization Group (MCRG) two-lattice matching technique.
  Two-lattice matching produces a step-scaling function $s_b$; for $N_f = 8$ we find $s_b$ to be clearly non-zero from weak coupling until the onset of strong-coupling lattice artifacts.
  For $N_f = 12$, in contrast, the step-scaling function exhibits a zero that indicates the presence of an infrared fixed point.
  To carry out MCRG two-lattice matching, we optimize the RG blocking transformation, which produces a step-scaling function that does not correspond to a unique RG $\beta$ function.
  We propose to address this issue by using the Wilson flow to eliminate the need for optimization of the RG blocking transformation, and present some preliminary results produced by this approach.
}

\FullConference{The 30th International Symposium on Lattice Field Theory \\ June 24--29, 2012 \\ Cairns, Australia}
% ------------------------------------------------------------------



% ------------------------------------------------------------------
\begin{document}
\section{Introduction and overview of methods \& results} % In progress
In recent years, many groups have initiated lattice investigations of strongly-coupled gauge--fermion systems beyond QCD.
While the ultimate goal of these efforts is to explore potential new physics beyond the standard model, an essential step is to improve our theoretical understanding of the basic properties of these non-perturbative systems.
Here we study the renormalization group properties of SU(3) gauge theories with $N_f = 8$ and 12 nearly-massless fermions in the fundamental representation, through the Monte Carlo Renormalization Group (MCRG) two-lattice matching technique.
This is one of several complementary analyses we are currently carrying out, two more of which (investigating Dirac eigenmode scaling and finite-temperature transitions) are discussed in other contributions to these proceedings~\cite{Hasenfratz:2012fp, Schaich:2012fr}.
Recent references on SU(3) gauge theories with $N_f = 8$ and 12 include~\cite{Fodor:2012uw, Fodor:2012et, Aoki:2012eq, Deuzeman:2012ee, Lin:2012iw}; earlier works are reviewed in \refcite{Giedt:2012LAT}.

In Refs.~\cite{Hasenfratz:2011xn, Hasenfratz:2011np}, one of us studied MCRG two-lattice matching for the 12-flavor system with nHYP-smeared staggered actions very similar to those we use here.
Our gauge actions include both fundamental and adjoint plaquette terms, with coefficients related by $\be_A = -0.25\be_F$.
The negative adjoint plaquette term lets us avoid a well-known spurious ultraviolet fixed point caused by lattice artifacts, and implies $\be_F = 12 / g^2$ at the perturbative level.
In our fermion actions, we use nHYP smearing parameters $(0.5, 0.5, 0.4)$, instead of the $(0.75, 0.6, 0.3)$ used by Refs.~\cite{Hasenfratz:2011xn, Hasenfratz:2011np}.
By changing the nHYP smearing parameters in this way, we can access stronger couplings without encountering numerical problems.
At such strong couplings, for both $N_f = 8$ and $N_f = 12$ we observe a lattice phase in which the single-site shift symmetry (``$S^4$'') of the staggered action is spontaneously broken (``$\Sb$'')~\cite{Cheng:2011ic, Schaich:2012fr}.\footnote{\refcite{Deuzeman:2012ee} recently interpreted the \Sb lattice phase in terms of relevant next-to-nearest neighbor interactions.}
We avoid the \Sb lattice phase in this work.

In the next section, we review how the MCRG two-lattice matching technique determines the step-scaling function in the bare parameter space, $s_b$.
Although working entirely in the bare parameter space would be disadvantageous if our aim were to produce renormalized phenomenological predictions for comparison with experiment, our current explorations of the phase structures of the 8- and 12-flavor systems benefit from this fully non-perturbative RG approach, especially for relatively strong couplings.

A limitation of MCRG two-lattice matching is the need to optimize the RG blocking transformation in order to determine $s_b$ for each lattice coupling $\be_F$.
As we explain in detail below, this optimization forces us to probe a different renormalization scheme for each $\be_F$, so that the bare step-scaling function we obtain is a composite of many different (discrete) RG \be functions.
While the existence of an infrared fixed point (IRFP) is physical (scheme-independent), the coupling at which it is located depends on the choice of renormalization scheme.
Thus, the 12-flavor results we present in \secref{sec:MCRGresults}, which show $s_b = 0$ for a wide range of $\be_F$, are consistent with the observation of an IRFP reported by Refs.~\cite{Hasenfratz:2011xn, Hasenfratz:2011np}.
In contrast, our 8-flavor results for $s_b$ are significantly different from zero for $\be_F$ ranging from relatively weak couplings to the \Sb lattice phase.

In addition to considering the traditional MCRG two-lattice matching technique, in \secref{sec:WMCRG} we also propose a new, improved procedure that predicts a bare step-scaling function corresponding to a unique discrete \be function.
This improved procedure involves applying the Wilson flow to the lattice system, and optimizing the flow time $t$ for fixed RG blocking transformation.
We present some promising but preliminary results of this approach in \secref{sec:WMCRGresults}.
\TODO{We expect that in the future this Wilson-flowed MCRG technique will shed light on the comparison of different studies, and may also be used to explore scaling violations and to study scheme transformation in the vicinity of an IRFP.}
% ------------------------------------------------------------------



% ------------------------------------------------------------------
\section{Two-lattice matching procedures} % In progress
Two-lattice matching is most easily described in the context of confining systems, where it locates pairs of couplings $(\be_F, \be_F')$ for which lattice correlation lengths obey $\xi(\be_F) = 2\xi(\be_F')$.
We proceed by repeatedly applying RG blocking transformations (with scale factor $s = 2$) to lattices of volume $24^3\X48$, $12^3\X24$ and $6^3\X12$.\footnote{We are currently generating larger lattices up to $32^3\X64$, which will permit additional consistency checks.}
Under RG blocking on the $m = 0$ critical surface, the system flows toward the renormalized trajectory in irrelevant directions, and along it in the relevant direction.
By blocking the larger lattices (with $\be_F$) $n_b$ times and the smaller lattices (with $\be_F'$) only $n_b - 1$ times, we obtain blocked systems with the same lattice volume.
If these blocked systems have both flowed to the same point on the renormalized trajectory, then we can conclude that $\xi(\be_F) = 2\xi(\be_F')$ on the unblocked systems, as desired.

We determine whether the blocked systems have flowed to the same point on the renormalized trajectory by matching several short-range gauge observables: the plaquette and all three six-link loops, with the $2\X2$ eight-link loop monitored as a check.
For a given $\be_F$, each observable may predict a different $\De\be_F \equiv \be_F - \be_F'$.
The spread in these results is a systematic error that dominates our uncertainties.

In the basin of attraction of an IRFP, correlation lengths diverge, and RG flows may not have a relevant direction that would define a renormalized trajectory.
However, the matching described above can still be carried out, and interpreted \TODO{in the following way\dots}

Since we can only block $24^3\X48$ lattices $n_b = 3$ times, we must optimize two-lattice matching by requiring that consecutive RG blocking steps yield the same $\De\be$.
Optimization is required in order to identify $\De\be_F$ with the bare step-scaling function $s_b$.
In the following two subsections, we describe two different ways to optimize two-lattice matching: the traditional technique optimizes the RG blocking transformation, while the new method we propose in \secref{sec:WMCRG} instead applies the Wilson flow to the lattice system and optimizes the flow time $t$.

First, we note that finite-volume effects are minimized by carrying out the optimization on blocked lattices of the same volume, which was reported by \refcite{Hasenfratz:2011xn}.
That is, we should compare $\De\be_F$ from matching $12^3\X24 \to 3^3\X6$ vs.\ $6^3\X12 \to 3^3\X6$ with that from matching $24^3\X48 \to 3^3\X6$ vs.\ $12^3\X24 \to 3^3\X6$.
\TODO{However, the error introduced by matching $24^3\X48 \to 6^3\X12$ vs.\ $12^3\X24 \to 3^3\X6$ instead of generating $6^3\X12$ lattices is not severe, and our preliminary results in \secref{sec:WMCRGresults}} \TODO{will use this approach.}
% ------------------------------------------------------------------



% ------------------------------------------------------------------
\subsection{Traditional MCRG} % In progress
Using the same technique as Refs.~\cite{Hasenfratz:2011xn, Hasenfratz:2011np}, we apply two sequential HYP smearings with parameters $(\al, 0.2, 0.2)$, and optimize the RG blocking transformation by tuning \al as shown in the left panel of \fig{fig:opt}.
Qualitatively, this optimization finds the renormalization scheme \TODO{for which the renormalized trajectory passes as close as possible to the lattice system with coupling $\be_F$.
Otherwise, residual flows in irrelevant directions could distort the results: this is the reason $\De\be$ changes with \al in \fig{fig:opt}.}

The downside of optimizing the RG blocking transformation in this manner is that we have to use a different renormalization scheme for each $\be_F$.
As a result, the bare step-scaling function we obtain is a composite of many different discrete RG \be functions.

\begin{figure}[htpb]
  \centering
  \includegraphics[width=0.45\linewidth]{optimization.pdf}\hfill
  \includegraphics[width=0.45\linewidth]{t_optimization.pdf}
  \caption{Optimization of the RG blocking transformation for $N_f = 12$ with $\be = 4.0$.  \TODO{Left: \dots
  Right: \dots}
  In both cases the red points come from matching \TODO{the plaquette for $12^3\X24 \to 3^3\X6$ vs.\ $6^3\X12 \to 3^3\X6$, while the blue points come from $24^3\X48 \to 3^3\X6$ vs.\ $12^3\X24 \to 3^3\X6$}.}
  \label{fig:opt}
\end{figure}
% ------------------------------------------------------------------



% ------------------------------------------------------------------
\subsection{\label{sec:WMCRG}Wilson-flowed MCRG} % In progress
As an alternative to optimizing the RG blocking transformation, and thus changing the renormalization scheme at each couplings $\be_F$, here we propose to \TODO{use the Wilson flow to move the lattice system as close as possible to the renormalized trajectory for a fixed renormalization scheme.}

The Wilson flow is a form of continuous smearing~\cite{Narayanan:2006rf} that \refcite{Luscher:2010iy} perturbatively related to the renormalized coupling in the \MSbar scheme.
Refs.~\cite{Fodor:2012td, Fodor:2012qh} recently adapted this scheme and used it to compute a step-scaling function in a way similar to Schr\"odinger functional methods.
While this approach appears very promising, we will only use the Wilson flow as a continuous smearing that removes UV fluctuations and thereby move the lattice system closer to the renormalized trajectory.
It is important to note that the Wilson flow moves the system along a surface of constant correlation lengths and lattice spacing; it is not a renormalization group transformation.

After applying the Wilson flow for a flow time $t$, we carry out MCRG two-lattice matching without having to optimize the RG blocking transformation.
The step-scaling function we obtain as a result is defined in the bare parameter space, completely independent of perturbation theory.
To determine when the Wilson flow has moved the lattice system \TODO{as close as possible to the renormalized trajectory,} we optimize the flow time $t$ for fixed RG blocking transformation.
That is, we measure $\De\be$ after various $t$ and require that consecutive RG blocking steps yield the same $\De\be$.

\TODO{Explore different RG blocking transformations/\be functions\dots \\
Around the perturbative gaussian fixed point, can use to check scaling violations\dots \\
At stronger couplings, may non-perturbatively investigate scheme transformations in the vicinity of the 12-flavor IRFP, an issue explored in perturbation theory by \refcite{Ryttov:2012nt}\dots}
% ------------------------------------------------------------------



% ------------------------------------------------------------------
\section{Results} % In progress
\subsection{\label{sec:MCRGresults}Traditional MCRG} % In progress
Our results for the bare step-scaling function $s_b$ from traditional MCRG two-lattice matching are shown in \fig{fig:MCRG}.
While our 8-flavor results for $s_b$ are significantly different from zero for all couplings we can explore, for $N_f = 12$ we find $s_b = 0$ for a broad range $4 \leq \be_F \leq 7$.
In both cases, we encounter the \Sb lattice phase at strong coupling, where we cannot perform matching.
At weaker couplings (larger $\be_F$), the step-scaling functions approach the perturbative predictions, $s_b \approx 0.6$ for $N_f = 8$ and $s_b \approx 0.3$ for $N_f = 12$.

\begin{figure}[htpb]
  \includegraphics[width=0.45\linewidth]{8flavor.pdf}\hfill
  \includegraphics[width=0.45\linewidth]{12flavor.pdf}
  \caption{Results for the step-scaling function $s_b$ from MCRG two-lattice matching with $24^3\X48$, $12^3\X24$ and $6^3\X12$ lattice volumes and the RG blocking transformation defined in the text, for $N_f = 8$ (left) and $N_f = 12$ (right).  In both cases we use fermion masses $m = 0.0025$ on $24^3\X48$, $m = 0.01$ on $12^3\X24$ and $m = 0.02$ on $6^3\X12$.  The blue dashed line is the perturbative prediction for $s_b$ at asymptotically weak coupling.}
  \label{fig:MCRG}
\end{figure}

To interpret our 12-flavor results, it is crucial to recall that our optimization of the RG blocking transformation means that we use a different renormalization scheme for each coupling $\be_F$.
If an IRFP exists, different renormalization schemes may locate it at a different coupling.
Specifically, our results imply that optimizing the RG blocking transformation selects a renormalization scheme with a fixed point at $\be_F$, so that $s_b = 0$.
While our results are consistent with the observation of a 12-flavor IRFP reported by Refs.~\cite{Hasenfratz:2011xn, Hasenfratz:2011np}, it is potentially interesting that we do not observe backward flow ($s_b < 0$) in this study.
% ------------------------------------------------------------------



% ------------------------------------------------------------------
\subsection{\label{sec:WMCRGresults}Wilson-flowed MCRG} % In progress
\TODO{Results\dots}
% ------------------------------------------------------------------



% ------------------------------------------------------------------
\section*{Acknowledgments} % Draft complete
We thank D\'aniel N\'ogr\'adi for helpful comments on the Wilson flow.
This research was partially supported by the U.S.~Department of Energy (DOE) through Grant No.~DE-FG02-04ER41290 (A.~C., A.~H.\ and D.~S.), and by the DOE Office of Science Graduate Fellowship (SCGF) Program made possible by the American Recovery and Reinvestment Act of 2009 and administered by the Oak Ridge Institute for Science and Education managed by Oak Ridge Associated Universities under Contract No.~DE-AC05-06OR23100.
Our code is based in part on the MILC Collaboration's public lattice gauge theory software,\footnote{\texttt{http://www.physics.utah.edu/$\sim$detar/milc/}} and on the code distributed with \refcite{Borsanyi:2012zs}.
Numerical calculations were carried out on the HEP-TH and Janus clusters at the University of Colorado, the latter supported by National Science Foundation (NSF) Grant No.~CNS-0821794; at Fermilab under the auspices of USQCD supported by the DOE SciDAC program; and at the San Diego Computing Center through XSEDE supported by NSF Grant No.~OCI-1053575.
% ------------------------------------------------------------------



% ------------------------------------------------------------------
\bibliographystyle{utphys}
\bibliography{pos2012}
\end{document}
% ------------------------------------------------------------------
